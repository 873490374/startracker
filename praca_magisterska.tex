\documentclass[12pt,a4paper,oneside]{article}
\usepackage[utf8]{inputenc}
\usepackage[english]{babel}
\usepackage{amsmath}
\usepackage{amsfonts}
\usepackage{fancyhdr}
\usepackage{graphicx}
\graphicspath{ {images/} }
\usepackage{amssymb}
\usepackage{bm}
\usepackage[intoc]{nomencl}
%\usepackage[authoryear,round,longnamesfirst]{natbib}

\setlength\headheight{15pt}
\enlargethispage*{0.7\baselineskip}


\pagestyle{fancy}
\fancyhf{}
\rhead{MICHALSKI Sz.}
\lhead{Star-tracker for Cubesat satellites}
%\lhead{Star-tracker dla Cubesat'ów}
\rfoot{\thepage}

\author{Szymon MICHALSKI}
\title{PROGRAM STAR-TRACKER DLA SATELITÓW TYPU CUBE-SAT}

\makenomenclature

\begin{document}

\begin{titlepage}
	\centering

	INSTITUTE OF CONTROL AND COMPUTATION ENGINEERING\par
	FACULTY OF ELECTRONICS AND INFORMATION TECHNOLOGY\par
	WARSAW UNIVERSITY OF TECHNOLOGY\par
%	INSTYTUT AUTOMATYKI I INFORMATYKI STOSOWANEJ\par
%	WYDZIAŁ ELEKTRONIKI I TECHNIK INFORMATYJNYCH\par
%	POLITECHNIKA WARSZAWSKA\par
	\vspace{0.5cm}
	\includegraphics[scale=0.3]{logo_WEiTI.jpg}
	\hspace{1cm}
	\includegraphics[scale=0.2]{Logo-PW-duze.jpg}
	\hspace{1cm}
	\includegraphics[scale=1]{ia_600_600.png}
	\par
	\vspace{2cm}
	MASTER OF SCIENCE THESIS\par
%	MAGISTERSKA PRACA DYPLOMOWA\par
	\vspace{0.2cm}
	{\huge STAR-TRACKER PROGRAM FOR CUBESAT SATELLITES\par}
%	{\huge PROGRAM STAR-TRACKER DLA SATELITÓW TYPU CUBE-SAT\par}
	\vspace{0.2cm}
	{\large Szymon MICHALSKI\par}
	\vspace{3cm}
	\begin{flushright}
	Supervisor:\par
%	Promotor:\par
	prof. dr hab. inż. Ryszard Romaniuk\par
	\end{flushright}
	\vspace{6cm}
	{\large Warszawa 2016\par}
\end{titlepage}

\pagenumbering{arabic}
\setcounter{page}{2}

\tableofcontents{toc}

\newpage
\setlength{\parindent}{1cm}
\setlength{\parskip}{\baselineskip}%


\nomenclature{$c$}{Speed of light in a vacuum inertial frame}
\nomenclature{$h$}{Planck constant}
\printnomenclature

\newpage
%http://ssl.mit.edu/publications/theses/SM-2006-HuffmanKara.pdf
\cite{jenssen2011comparison}
\cite{valenti2015keeping}
\cite{delabie2012highly}
\cite{jalabert2011optimization}
\cite{felikson2011orbit}
\cite{knutson2012fast}
\cite{rose2003star}
\cite{mortari2002starnav}


\section{Introduction}
\subsection{Motivation}
The goal of this work is to make fully operational star-tracker program, that could be used on Cubesat satellites. Such program could be used on space missions and could start Polish state-of-the-art technology in growing space technology sector.

\subsection{Outline of thesis}



This thesis consists of several chapters. Here they are shortly summarized:\par
\setlength{\parindent}{0cm}
\textbf{Chapter 1} serves as introduction to this thesis and describest the motivation and goal of this work. It also describes the background of the topic.\par
\textbf{Chapter 2} describes all the important foundations for the fully understanding given work.\par
\textbf{Chapter 3} is the main part of this thesis. It describes how the star-tracker program works and goes through detailed comparison of different approaches.\par
\textbf{Chapter 4} describes the created prototype of star-tracker in Python language.\par
\textbf{Chapter 5} talks about the implementation of star-tracker on the existing prototype of on-board computer.\par
\textbf{Chapter 6} describes how the finished program is performing.\par
\textbf{Chapter 7} contains conclusons about this work and created star-tracker program.\par

\setlength{\parindent}{1cm}

\subsection{Cubesat}

%https://en.wikipedia.org/wiki/CubeSat#cite_note-SpaceNews-2015-06-08-1

Cubesat was designed on CalPoly in 1999\cite{heidt2000cubesat}.
Dimensions of satellite are measured in units. Each unit (often described simply as u) can be 10x10x10cm and can weight up to 1.33 kg. Satellites can be 1u, 2u, 3u, 6u or even 12u.

Such small satellites are suspectible to noise from densly packed electronics.

Zdjecie Cubesata

CubeSat missions, goals, what can they be and are used for? Why is it innovative and important?

\subsection{Means of attitude estimation}

There exist many different types of attitude estimation: sun sensors, star-trackers, magnetometers, etc. However star-tracker gives the best possible accuracy for nowadays and is not suspectible to electrical nor magnetic noise.

\cite{larson1992space}
\begin{table}[ht]

\begin{tabular}{ccc}
Sensor & Accuracy & Characteristics and Applicability \\ 
Magnetometers & 1.0o (5000km alt) 5.0 (200 km alt) & Attitude measured relative to Earth’s local magnetic field. Magnetic field uncertainties and variability dominate accuracy. Usable only below ~ 6,000 km. \\ 
Earth sensors & 0.05 (GEO) 0.1 (LEO) & Horizon uncertainties dominate accuracy. Highly accurate units use scanning. \\ 
Sun sensors & 0.01 & Typical field of view +-30 \\ 
Star sensors & 2 arc-sec & Typical field of view +-6 \\ 
Gyroscopes & 0.001 deg/hr & Normal use involves periodically resetting reference. \\ 
Directional antennas & 0.01 to 0.5 & Typically 1 of the antenna beamwidth

\end{tabular} 
\caption{Sensor Accuracy Ranges. Adapted from \cite{hall2003spacecraft}}

\end{table}
\cite{lima2000comparison}

\subsection{On-board computer}
This section will describe the on-board computer which was done as part of other thesis.

\newpage
\section{Preliminaries}
\subsection{Coordinate frames}
\subsubsection{ECI frame}
\subsubsection{ECEF frame}
\subsubsection{NED frame}
\subsubsection{BODY frame}
\subsection{Space environment}
\subsection{Attitude representations}
\subsubsection{Euler angles}
\subsubsection{Quaternions}
\subsection{Quaternion properties}
\subsubsection{Advantages of quaternions}
\subsubsection{Multiplication of quaternions}
\subsubsection{Quaternions and rotations}
\subsection{Wahba's problem}
\cite{wahba1965least}

\begin{equation}
\sum_j^n ||r_j - Mb_j||
\end{equation}
\subsection{Cholesky factorization}
\subsection{Lyapunov analysis}

\newpage
\section{Star-tracker program}
\cite{ju2003overview}\par
Generally star-tracker is divided into three main parts\cite{6187242}:
\begin{itemize}
\item recogiting stars on the image and converting the data into list of star vectors by calculating star centroids;
\item identyfing which star vector represents which real star in catalogue. This is done by comparing star vectors from the image with data in star catalogue, which is generated before space mission;
\item estimating the attitude by calculating the displacement between two frames.
\end{itemize}
\subsection{Centroid - start recognition}
\cite{samaan2002predictive}

Due to limitations of camera there exists necessity of calculating star centroids. Each camera converts image into photo divided by pixels. As it is necessary to have high precision of star coordinates, the pixel accuracy is not enough. Subpixel accuracy is needed. Typically it is done by defocusing the lens of the camera and calculating the lumosity of all pixels around the lightest ones. The idea of how to calculate such centroids is adapted from\cite{6187242}.

If FOV is too small, one star will be considered by program as few stars, and if FOV is too large, few stars placed near each other will be considered as one star. Calculating star centroids is tradeoff between counting few stars as one and counting one star as a few. It seems however that it is worse to count one star as few than few stars as one.

\begin{equation}
x_{start} = x - \frac{a_{ROI} - 1}{2}
\end{equation}
\begin{equation}
y_{start} = y - \frac{a_{ROI} - 1}{2}
\end{equation}
\begin{equation}
x_{end} = x_{start} + a_{ROI}
\end{equation}
\begin{equation}
y_{end} = y_{start} + a_{ROI}
\end{equation}
\begin{equation}
I_{bottom} = \sum_{i=1}^{x_{end}-1} I(i, y_{start})
\end{equation}
\begin{equation}
I_{top} = \sum_{i=2}^{x_{end}} I(i, y_{end})
\end{equation}
\begin{equation}
I_{left} = \sum_{j=1}^{y_{end}-1} I(x_{start}, j)
\end{equation}
\begin{equation}
I_{right} = \sum_{j=2}^{y_{end}} I(x_{start}, j)
\end{equation}
\begin{equation}
I_{border} = \frac{I_{top} + I_{bottom} + I_{left} + I_{right}}{4(a_{ROI} - 1)}
\end{equation}

\begin{equation}
\tilde{I}(x,y) = I(x,y) - I_{border}
\end{equation}

\begin{equation}
B = \sum_{i=x_{start}+1}^{x_{end}-1}\sum_{j=y_{start}+1}^{y_{end}-1}\tilde{I}(i,j)
\end{equation}
\begin{equation}
x_{CM} = \sum_{i=x_{start}+1}^{x_{end}-1}\sum_{j=y_{start}+1}^{y_{end}-1}\frac{i \times \tilde{I}(i,j)}{B}
\end{equation}
\begin{equation}
x_{CM} = \sum_{i=x_{start}+1}^{x_{end}-1}\sum_{j=y_{start}+1}^{y_{end}-1}\frac{j \times \tilde{I}(i,j)}{B}
\end{equation}

\begin{equation}
u = \frac{
\begin{bmatrix}
\mu x_{CM} & \mu y_{CM} & f
\end{bmatrix}
^T}
{||
\begin{bmatrix}
\mu x_{CM} & \mu y_{CM} & f
\end{bmatrix}
||}
\end{equation}

\subsection{Star identification}
all \cite{spratling2009survey}\par
Brightness Independent 4-Star Matching Algorithm for Lost-in-Space 3-Axis Attitude Acquisition\cite{dong2006brightness} \par
SP-Search: A New Algorithm for Star Pattern Recognition \cite{mortari1999sp} \par
Star Identification using Neural networks \cite{miri2012star} \cite{lindbladstar} \par
Star pattern recognition using neural networks \cite{li2003star} \par


\subsubsection{Angle Matching}
% https://books.google.pl/books?id=GtzzpUN8VEoC&pg=PA259&lpg=PA259&dq=Gottlieb+Star+Identification+Techniques+1978&source=bl&ots=6_y_cnQUJe&sig=TBkqnYn41hOWcswFdt1a1fghNcc&hl=en&sa=X&ved=0ahUKEwjuiriUuNnOAhWE_iwKHX46AIgQ6AEIIDAA#v=onepage&q=Gottlieb%20Star%20Identification%20Techniques%201978&f=false
\cite{gottlieb1978star}
\begin{equation}
\theta = \cos^{-1}(\bm{r}_1 \cdot \bm{r}_2)
\end{equation}
\begin{equation}
\bm{b}_i = A\bm{r}_i
\end{equation}
\begin{equation}
\tilde{\bm{b}}_i = A\bm{r}_i + \bm{v}_i, \hspace{0.5cm} \bm{v}_i^TA\bm{r}_i = 0
\end{equation}
\begin{subequations}
\begin{align*}
E\left\{\bm{v}_i\right\} &= 0 \\
E\left\{\bm{v}_i\bm{v}_i^T\right\} &= \sigma_i^2 [\bm{I} - (A\bm{r}_i)(A\bm{r}_i)^T]
\end{align*}
\end{subequations}
\begin{equation}
\bm{b}_i^T\bm{b}_j = \bm{r}_i^TA^TA\bm{r}_j = \bm{r}_i^T\bm{r}_j
\end{equation}
\begin{subequations}
\begin{align*}
\tilde{\bm{b}}_i = A\bm{r}_i + \bm{v}_i\\
\tilde{\bm{b}}_j = A\bm{r}_j + \bm{v}_j
\end{align*}
\end{subequations}
\begin{equation}
z \equiv \tilde{\bm{b}}_i^T\tilde{\bm{b}}_j = \bm{r}_i^T\bm{r}_j + \bm{r}_i^TA^T\bm{v}_J + \bm{r}_j^TA^T\bm{v}_i + \bm{v}_i^T\bm{v}_j
\end{equation}
\begin{equation}
E\left\{z\right\} = \bm{r}_i^T\bm{r}_j
\end{equation}
\begin{equation}
p \equiv z - E\left\{z\right\} = \bm{r}_i^TA^T\bm{v}_J + \bm{r}_j^TA^T\bm{v}_i + \bm{v}_i^T\bm{v}_j
\end{equation}
\begin{equation}
\sigma_p^2 \equiv E\left\{p\right\} = \bm{r}_1^TA^TR_2A\bm{r}_1 + \bm{r}_2^TA^TR_aA\bm{r}_2 + Trace(R_1R_2) = Trace(A\bm{r}_1\bm{r}_1^TR_2) + Trace(A\bm{r}_2\bm{r}_2^TR_1) + Trace(R_1R_2)
\end{equation}
\subsubsection{Spherical Triangle Matching}
\cite{cole2004fast}

\begin{equation}
A = 4\tan^{-1}\sqrt{\tan\frac{s}{2}\tan\frac{s-a}{2}\tan\frac{s-b}{2}\tan\frac{s-c}{2}}
\end{equation}
\begin{subequations}
\begin{align*}
s = \frac{1}{2}(a + b + c) \\
a = \cos^{-1} \bigg(\frac{b_1 \cdot b_2}{|b_1||b_2|}\bigg) \\
b = \cos^{-1} \bigg(\frac{b_2 \cdot b_3}{|b_2||b_3|}\bigg) \\
c = \cos^{-1} \bigg(\frac{b_3 \cdot b_1}{|b_3||b_1|}\bigg) 
\end{align*}
\end{subequations}
\begin{equation}
I_p = \sum\theta^2dA
\end{equation}
\subsubsection{Planar Triangle}
\cite{cole2006fast}\par

\begin{equation}
s = \frac{1}{2}(a + b + c)
\end{equation}

\begin{equation}
a = ||\bm{u_p} - \bm{u_q}||
\end{equation}
\begin{equation}
b = ||\bm{u_q} - \bm{u_r}||
\end{equation}
\begin{equation}
c = ||\bm{u_p} - \bm{u_r}||
\end{equation}

\begin{equation}
A = \sqrt{s(s-a)(s-b)(s-c)}
\end{equation}

\begin{equation}
J = A\frac{(a^2 + b^2 + c^2}{36}
\end{equation}

Derivatives
\begin{equation}
H = \begin{bmatrix}
\bm{h}_1^T & \bm{h}_2^T & \bm{h}_3^T
\end{bmatrix}
\end{equation}
\begin{subequations}
\begin{align*}
\bm{h}_1^T \equiv \frac{\delta A}{\delta a}\frac{\delta a}{\delta\bm{b}_1} + \frac{\delta A}{\delta c}\frac{\delta c}{\delta\bm{b}_1} \\
\bm{h}_2^T \equiv \frac{\delta A}{\delta a}\frac{\delta a}{\delta\bm{b}_2} + \frac{\delta A}{\delta b}\frac{\delta b}{\delta\bm{b}_2}  \\
\bm{h}_3^T \equiv \frac{\delta A}{\delta b}\frac{\delta b}{\delta\bm{b}_3} + \frac{\delta A}{\delta c}\frac{\delta c}{\delta\bm{b}_3} 
\end{align*}
\end{subequations}

\begin{subequations}
\begin{align*}
\frac{\delta A}{\delta a} = \frac{u_1 - u_2 + u_3 + u_4}{4A} \\
\frac{\delta A}{\delta b} = \frac{u_1 + u_2 - u_3 + u_4}{4A} \\
\frac{\delta A}{\delta c} = \frac{u_1 + u_2 + u_3 - u_4}{4A}
\end{align*}
\end{subequations}

\begin{subequations}
\begin{align*}
u_1 = (s - a)(s - b)(s - c) \\
u_2 = s(s - b)(s - c) \\
u_3 = s(s - a)(s - c) \\
u_4 = s(s - a)(s - b) \\
\end{align*}
\end{subequations}

\begin{subequations}
\begin{align*}
\frac{\delta a}{\delta \bm{b}_1} = (\bm{b}_1 - \bm{b}_2)^T /a, \hspace{0.5cm} \frac{\delta a}{\delta \bm{b}_2} = -\frac{\delta a}{\delta \bm{b}_1} \\
\frac{\delta b}{\delta \bm{b}_2} = (\bm{b}_2 - \bm{b}_3)^T /b, \hspace{0.5cm} \frac{\delta b}{\delta \bm{b}_3} = -\frac{\delta b}{\delta \bm{b}_2} \\
\frac{\delta c}{\delta \bm{b}_1} = (\bm{b}_1 - \bm{b}_3)^T /c, \hspace{0.5cm} \frac{\delta c}{\delta \bm{b}_3} = -\frac{\delta c}{\delta \bm{b}_1}
\end{align*}
\end{subequations}

\begin{equation}
\sigma_A^2 = HRH^T
\end{equation}

\begin{equation}
R \equiv \begin{bmatrix}
R_1 & 0_{3x3} & 0_{3x3} \\
0_{3x3} & R_2 & 0_{3x3} \\
0_{3x3} & 0_{3x3} & R_3
\end{bmatrix}
\end{equation}

Polar Moment

\begin{equation}
\bar{H} = \begin{bmatrix}
\bar{\bm{h}}_1^T & \bar{\bm{h}}_2^T & \bar{\bm{h}}_3^T
\end{bmatrix}
\end{equation}

\begin{subequations}
\begin{align*}
\bar{\bm{h}}_1^T \equiv \frac{\delta J}{\delta a}\frac{\delta a}{\delta\bm{b}_1} + \frac{\delta J}{\delta c}\frac{\delta c}{\delta\bm{b}_1} + \frac{\delta J}{\delta A}\bm{h}_1^T\\
\bar{\bm{h}}_2^T \equiv \frac{\delta J}{\delta a}\frac{\delta a}{\delta\bm{b}_2} + \frac{\delta J}{\delta b}\frac{\delta b}{\delta\bm{b}_2} + \frac{\delta J}{\delta A}\bm{h}_2^T \\
\bar{\bm{h}}_3^T \equiv \frac{\delta J}{\delta b}\frac{\delta b}{\delta\bm{b}_3} + \frac{\delta J}{\delta c}\frac{\delta c}{\delta\bm{b}_3} + \frac{\delta J}{\delta A}\bm{h}_3^T
\end{align*}
\end{subequations}

\begin{subequations}
\begin{align*}
\frac{\delta J}{\delta a} = A a/18, \hspace{0.5cm} \frac{\delta J}{\delta a} = A b/18, \hspace{0.5cm} \frac{\delta J}{\delta a} = A c/18 \\
\frac{\delta J}{\delta A} = (a^2 + b^2 + c^2)/36
\end{align*}
\end{subequations}

\begin{equation}
\sigma_J^2 = \bar{H}R\bar{H}^T
\end{equation}

\subsubsection{Pyramid}
\cite{mortari2004pyramid}\par

\subsubsection{Rate Matching}
\cite{rao2005fast} 
\subsubsection{Voting}
\cite{kolomenkin2008geometric} \par
\subsubsection{Grid}
%http://dsp.ucsd.edu/~kreutz/Publications/Padgett1997.pdf
\cite{padgett1997grid}
\subsection{Star-catalogue and searching for matching stars}

\subsubsection{Star Catalogue Generation}
\begin{equation}
\bm{u} = \begin{bmatrix}
\cos \alpha \cos \delta \\
\sin \alpha \cos \delta \\
\sin \delta
\end{bmatrix}
\end{equation}
\begin{equation}
m_i \leq m_{max}
\end{equation}
\begin{equation}
m_j \leq m_{max}
\end{equation}
\begin{equation}
\bm{u_a^T u_b} \geq \cos \theta_{FOV}
\end{equation}

\subsubsection{Candidate Matching}
\subsubsection{Result Verification}
\subsubsection{k-vector}

\cite{mortari2013k}\par
\cite{mortari1996fast}\par
\cite{mortari2000k}\par

\subsection{Attitude Determination}
\cite{jenssen2011comparison}\par
AIM (Attitude estimation using Image Matching)\cite{delabie2012highly}\par
all \cite{hall2003spacecraft} \cite{markley1999estimate}
\subsubsection{The Predictive Attitude Determination Algorithm ?}
\cite{park2006attitude}
\subsubsection{q-method}
\subsubsection{QUEST}
improvement to quest implementation \cite{RIS_1} \par
kallman filtering \cite{shuster1990kalman}
\subsubsection{TRIAD}
\subsubsection{The Fast Optimal Attitude Matrix}
\subsubsection{DCM (direction cosine matrix) - (Singular Value Decomposition?)}
\cite{juang2003efficient}
 and

\cite{6187242}
\begin{equation}
\bm{B = \sum_{i=1}^nb_ir_i^T}
\end{equation}
\begin{equation}
\bm{B = USV^T}
\end{equation}
\begin{equation}
\bm{U}_+ = \bm{U}\begin{bmatrix}
1 & 0 & 0 \\
0 & 1 & 0 \\
0 & 0 & det\bm{U}
\end{bmatrix}
\end{equation}
\begin{equation}
\bm{V}_+ = \bm{V}\begin{bmatrix}
1 & 0 & 0 \\
0 & 1 & 0 \\
0 & 0 & det\bm{V}
\end{bmatrix}
\end{equation}
\begin{equation}
\bm{A = U_+V_+^T}
\end{equation}
\newpage
\section{Prototype}
For now the following parts are finished in Python:
\begin{enumerate}
\item Centroiding
\item Planar Triangle Recognition with variations (nearly)
\item Pyramid alg
\item k-vector
\item QUEST (not started yet)
\end{enumerate}
Testing\par
\cite{kruijff2003star}
\newpage
\section{Complete program}

\newpage
\section{Testing of star-tracker}
\cite{RIS_0}

\newpage
\bibliographystyle{ieeetr}
\bibliography{bibfile}




\newpage

\addcontentsline{toc}{chapter}{List of Tables}
\listoftables

\newpage

\addcontentsline{toc}{chapter}{List of Figures}
\listoffigures

\newpage


Robot Learning Darmstadt
Problems with Euler Angles:
Not Unique: Many angles result in the same rotation
Hard to quantify differences between two Euler Angles
Unit-Quaternion
Solves the problems of singularities with the Euler Angles
Easier to compute differences of orientations
Important if we want to control the orientation of the end-effector
See Siciliano or Spong Textbook!

%https://en.wikipedia.org/wiki/Apparent_magnitude

Polar moment

\end{document}
